% !TeX root = ../../thesis.tex
\chapter*{Acknowledgment}                                  \label{ch:preface}

My favorite part of reading any PhD thesis is the acknowledgement section, as my close friends can attest. This section, in my opinion, is the most crucial part of any thesis, as it highlights the invaluable support that has made it possible. Throughout my PhD journey, I have grown tremendously in many ways, and I am grateful for the opportunities that I was provided with and the people who surrounded me during this chapter of my life, giving me their support, energy, presence, and love. As I reflect on the years spent working on this thesis, I realize that the experience would have been incomplete without the contributions of these awesome people. While I could easily write an entire thesis on these amazing experiences and individuals, I will keep it brief for the sake of brevity.

Before starting to write, I should appreciate the funding agencies and the Prosperos project, without which my PhD and all the things I’m going to talk about would have never existed. The Prosperos project was funded by the Interreg VA Flanders - The Netherlands program, CCI grant no. 2014TC16RFCB046 and by the Fund for Scientific Research Flanders (FWO), grant G085018N. Also, I acknowledge support from the European Research Council (ERC) under the European Union's Horizon 2020 research and innovation program, ERC CoG 772418. The computational resources and services used in this work were mostly provided by the VSC (Flemish Supercomputer Center), funded by the Research Foundation - Flanders (FWO) and the Flemish Government - department EWI. 

Moreover, my sincere acknowledgement goes to my examination committee members, Jos, Regine, Gabor, and Prof. Van der Biest for their valuable feedback for shaping a better thesis. Special thanks goes to Hans and Nele, which have been with me from the very beginning as part of my supervisory committee, from whom I have learned a lot.

Lies, where do I even begin? You have been an incredible supervisor to me throughout my scholarship, and I cannot thank you enough for the trust and freedom you gave me in this period. Being able to work on whatever I wanted to was truly a remarkable experience, and it's all thanks to you. I will be forever grateful for this opportunity and the impact it has had on my life. I've always admired you as a great scientist, and I have learned so much from you. Your support and guidance have been instrumental in shaping me into the researcher I am today. It may sound like a cliché, but it's true: I owe you a lot. Your influence on me will continue to inspire and guide me in every stage of my future career. Thank you, Lies, for being such an amazing mentor and support.

Fernando, or as they say, ``Dear Fernando'', you truly are a man for all seasons! Before coming to Belgium, I was fortunate enough to have some amazing close friends with whom I could share my life. But I never could have imagined that I would find such a remarkable friend in someone with such a different cultural background. And yet, it happened! Our friendship has been a source of relief and mindfulness for me, and I am so proud to call you my friend. You have been a constant source of support and encouragement, always there when I needed someone to talk to or simply hang out with. I am grateful to you for all the wonderful moments we have shared together. And let's not forget about our long discussions on technical and scientific matters, which I always find so stimulating. Although we may disagree on certain things (like the eternal debate of shared vs. static libraries or Ubuntu vs. Debian), I know that our friendship will always remain strong. Marz and I are both so lucky to have you in our lives, Fernando. Thank you for being such an incredible buddy.

\begin{flushright}
\fontspec{XB Zar}
\foreignlanguage{persian}
{
ماهور، سلطان السلاطین، استاد الاساتید، و بزرگ مرد بامعرفت لوون. ما خیلی دیر همدیگه رو کشف کردیم، شاید خیلی شانسی، ولی خب در این شک ندارم که آشنایی با تو از شانسهای بزرگ زندگی من بوده. من و ماه و تو باعث دوستی ما شد، و بعد شروع کردیم به ساختن خاطره‌های دلچسب بسیار. میخوای بگم یکمشو؟ پارک ستار، فیدبکینگ تو محله مولوی بروکسل، بحثهای ساعت 4 صبح تو کوچه، "ماهور، پیتزا واز گود؟"، آخ این ... اومد، رابرت مرتضوی، ماشین پیتزایی، تمرین بیسیست بند شب قبل اجرا رو شکمش، "مجی یه لا ماژور بگیر ببینم"، آهنگ‌های لویی پسند، روحیه بند، کریم، فعل‌های ماهَوی، تولید محتوا با استاد باراسا، نجات جنگل با پنج‌دری، کامیون بیسم، ساعت صفر تاریخ و کلی کلیدواژه (و شاید در واقع گل‌واژه) دیگه. در کنار همه اینا، من تا ابد ازت ممنونم بابت اینکه من رو راه دادی توی بندت، بابت اینکه کلی بهم اعتمادبنفس دادی تو نوازندگی، و کلا بابت همه چیزهایی که بهم یاد دادی توی موسیقی. دمت خیلی گرم پسر. اینم بگم که اگر حتی فقط یه روز از عمرم مونده باشه، مطمئنا برمیگردم و انتقامم رو تو تخته ازت میگیرم.
}
\end{flushright}

Laura, the little math girl, you truly are a booster that makes our office a more lovely place! Your warmth and kindness brought so much joy to our office, and I am lucky to know you. You have always been an endless source of support and positive energy for me, and by ``endless'', I truly mean it.  I cannot thank you enough for all that you have done for me. Working with you has always been a pleasure, especially during those high-pressure projects that require crisis management skills like the kindergarten project. And, I have learned a lot from you too, and this is not an exaggeration. But most importantly, I'm grateful to have you as a friend. Thank you, the little math girl, for being such an awesome friend.

Ales, you truly are a card magician with a mind full of DB hacks! Is this enough to describe you? Of course not. You have a kind soul and a great sense of humor that can make me laugh at any kind of shit I say. It's been an amazing experience getting to know you and realizing just how similar we are when it comes to jokes and making fun of everything. I was lucky enough to be on the same team as you in the reception party, which led to us becoming great friends. I'll never forget the evening we first got to know each other, as well as all the other fun times we've had together. Even though I'm not interested in alcohol (Marz: +10 points), those evenings will always be unforgettable for me. I hope that life has more mutual funny stories in store for us to share. And by the way, you should feel honored to be a friend of the healthiest man in Leuven! Thanks for being an amazing friend, Ales.

Jurgen, my favorite Belgian, my favorite European driver, and the best person to make an alliance with when it comes to beating Fernando in board games. You are one of the most creative people I have ever met in my life, and your boundless imagination and mental fantasy can make everything much funnier. I can't thank you enough for all the support and positive energy you have given me. As a famous Persian quote says, "if you want to get to know someone, travel with them", and our trip to Norway was a perfect example of this. Spending time with you was an unforgettable experience, and beside all the good memories we made, it allowed me to see just how wonderful of a person you truly are. Thank you for being my friend, Mr. Kersschot.

Satanik, you were my first friend when I arrived in this foreign land. I can't express how grateful I am for the warmth and kindness you showed me during those difficult initial months. I'll always cherish the memories of our conversations during those lunch breaks at Alma 3, where we were talking about everything from the power of the mind to the histories and languages of our respective countries. Your insights helped me understand our continent better, and I'll always remember the funny stories we shared about our cultures. Our lunch breaks eventually turned into quick 5-minute coffee breaks, for which you were the most accurate time control person I've ever met in my life. Thank you for being such an important part of my life in this new place.

\begin{flushright}
\fontspec{XB Zar}
\foreignlanguage{persian}
{
فهیمه، تو تا همیشه خیلی برای من ویژه خواهی موند، نه تنها بخاطر شخصیت خوب و حمایتگرت و همه لطف‌هایی که بهم داشتی در طی این سالها، بلکه به این دلیل که اولین نفری بودی که توی این سفر طولانی سر راه من قرار گرفت، و چقدر اون اولین دیدار عجیب بود و چقدر الان دور بنظر میرسه. سفارت بلژیک، و تعجب جفتمون از اینکه دقیقا داریم میریم یه بخش تو اون دانشگاه به اون بزرگی. و بعد از اون رگباری اتفاقات و خاطره‌ها و حماسه‌های پشت هم تا همین امروز. از اون شب بازی ایران اسپانیا که ویزای تو اومد بگیریم بیایم تا ده‌ها خاطره مشترک جذابی که چهارتایی داریم. و لذت تمام اون صحبت‌هایی که همیشه با هم داشتیم در مورد مشکلاتی که مشترکا داشتیم تجربه میکردیم توی بخش بیومکانیک، مخصوصا سال اول، و اون تکنیکی که استفاده میکردیم که بقیه نفهمن راجع به چی داریم حرف میزنیم، و اسم رمزها، مگه میشه اینا رو یه روزی یادم بره؟ فلفل، شادی، گلگلی، ملکه، پدرخوانده، موقشنگ، فندق، نارگیل. من تا همیشه بابت این دوستی و این حمایتگری ازت ممنون خواهم بود همشیره.
}
\end{flushright}

Pieter, it was truly a pleasure working with you during my time at KU Leuven. I never would have thought that the person who initially missed the first two introductory sessions of their master's thesis project would become one of my closest friends, but that's exactly what happened. Your brilliance as a researcher, musician, and friend is truly remarkable, and it didn't take long for me to realize that. I will always cherish our conversations about music, especially when we discovered that we both shared a love for a less-popular sub-genre of metal music. Although I was a bit disappointed when you showed interest in John Mayer later on, it didn't change how much I valued your friendship. Playing in the same band with you was an experience that I will never forget, from our wonderful jam sessions to the thrill of playing on stage together. I hope we can stay in touch and continue to make more great memories.

\begin{flushright}
\fontspec{XB Zar}
\foreignlanguage{persian}
{
احسان، یکی از حسابی ترین آدمهایی که میشناسم، و مرد مردان گروه لیز. از همون اولین باری که دیدمت، که شام خداحافظی وارون بود تو لیژ، مشخص بود که چه شخصیت جذابی هستی، آدم پخته، حمایتگر، و مشتی، از اینهایی که باهاشون بودن همیشه به آدم حس خوب میده. و گذر زمان هم اینو بشدت اثبات کرد. از اون اولین سفری که با هم رفتیم، همونی که چای میکس زعفرونی کلیدواژشه و ترجمه‌های من از تاکتیک‌هاتون باعث شد تیمتون ده دو ببازه، تا تموم اون شب‌هایی که خونتون چتر بودیم و نمیذاشتی بریم خونمون، همه و همش برای من کلی خاطره شیرینه. دوستی با تو، و صدالبته با نسترن، از اون دست دوستی‌هاییه که بهش افتخار میکنم. تو و نسترن الگوهایی هستین از بسیار بسیار کمیاب آدمهایی که سالهای زیادی خارج از ایران زندگی کردن ولی هنوز خوی آدم‌حسابی بودنشون رو حفظ کردن. و این چیزیه که همیشه من و مرضیه میگیم در موردتون وقتی میخوایم شما رو مثال بزنیم برای آدمهایی که نمیشناسنتون. دمتون خیلی گرم.
}
\end{flushright}

Michel, a German hip-hop lover, my favorite metalhead in the world, and my favorite concert buddy. You are truly a one-of-a-kind person, and I feel lucky to have had the chance to get to know you. When I first joined BMe, it was immediately clear to me that you were a special member of the team. Your inclusive and fun-loving personality made you stand out in the best way possible. You never failed to make me laugh, even during the most stressful times. I have so many great memories of the trips and concerts we went to together, and I will always cherish the moments we spent chatting and laughing at BMe. Even though the pandemic kept us apart, I know our friendship will continue to thrive. Thank you for being such an amazing friend and for all the wonderful memories you have given me.

\begin{flushright}
\fontspec{XB Zar}
\foreignlanguage{persian}
{
مجید، استاد صالحی، چی بگم در موردت؟ از کدوم بخشت بگم؟ بگم شاید تنها کسی که میشد باهاش در مورد جزئی‌ترین موضوعات فنی کامپیوتر ساعت‌ها حرف زد؟ نه خیلی نردی میشه. بگم کسی که باهاش سر اینکه کدوممون بیشتر مادر سلامتی‌ها هستیم رقابت داشتم؟ نه بابا چه کاریه. پس اینطوری میگم: یکی از پایه‌ترین آدمهایی که بهشون برخوردم تو زندگی، که چه موقع خوبی هم اتفاقا سر راهم قرار گرفت. سال اول که سال تنهایی و غربت سنگین بود برای خیلیامون. برای من که بود. و بعدش هم سالهای بعدی پشت هم. پسر، چقدر خاطره داریم و چقدر خوش گذروندیم. ولی میدونی؟ من اگر بخوام فقط یه خاطره رو انتخاب کنم که داشته باشم ازت برای همیشه، اون عصری رو انتخاب میکنم که در اوج ضیق وقت، برای رسیدن به عشرتی که دنبالش بودیم، سر از بروکسل دراووردیم. ولی خدایی، کاش به احترام این چندین سال رفاقت، بیای و اون حدود دویست یورو شرطهایی که باختی رو باهام صاف کنی که روز جزا سر راهت سبز نشم.
}
\end{flushright}

Myrna, I want to start by apologizing for how often I showed up at your place in Leuven. Looking back, I realize that I must have been quite annoying with my frequent visits, and I am sorry for that. However, I hope you know that I always enjoyed spending time with you, Mahoor, and now Mehr. Your warm hospitality and welcoming personality always made me feel at home, even when I was far from my own. I appreciate all the times you let me crash at your place, and I will always cherish the countless dinners you prepared for us with your wonderful cooking skills. Thank you for everything, Myrna.

\begin{flushright}
\fontspec{XB Zar}
\foreignlanguage{persian}
{
ایمان، استاد صباحی، مرد خوش آواز و خوش تحلیل من. مردی که همیشه با خودش لیست میبره برای خرید. و اسم بیشتر بندهای موسیقی براش آشناست. و تاکید داره که نام خانوادگی نوشته بشه روی امضا. و از برنج و تخم مرغ استفاده میکنه برای توضیح مفاهیم پیچیده علمی. و خوشمزه‌ترین ماهی‌های تاریخ رو میپزه. ایمان، من همیشه از صحبت کردن و همنشینی باهات لذت بردم، چون از اون دست آدمهایی هستی که خیلی راحت میشه باهات رفت به عمق، بدون اینکه تلاش خاصی نیاز باشه. یکی از بزرگترین نیازها و شاید تفریحات من حرف زدن و بحث کردنه، و تو از معدود افرادی بودی که همیشه این نیاز رو در من عمیقا ارضا کردی، و بخاطر همین بود که وقتی حرفامون شروع میشد بدون اینکه خودمون متوجه باشیم تقریبا همیشه خیلی راحت زمان از دستمون در میرفت و یهو میدیدیم خیلی بی اختیار چندین بار دور هورلی بیلز چرخیدیم. و داستان‌های من‌وماه‌وتو هم که دیگه هیچی، تکمیل کرد قشنگ همش رو. دمت گرم پسر، که صد البته در این زمینه باید ممنون باشم از محسن و مرضیه هم، که باعث این آشنایی شدن.
}
\end{flushright}

Jorge, I'll never forget the time we played a gig together in Gent and after the show, as we were descending the staircase on the back of the stage, you said that co-authoring a paper together was one thing, but playing on the same stage was a whole other level. That moment really captured our journey as friends. I'm grateful for all the amazing experiences we've had together, from our trips to playing music and even playing chess (especially the games we played in Mahoor’s car before and after the Gent gig).  Your warm support for me and TuxRiders on social media also means the world to me. Thank you so much buddy.

\begin{flushright}
\fontspec{XB Zar}
\foreignlanguage{persian}
{
آرزو، مگه میشه از لوون حرف زد و از تو نگفت؟ تا آخر عمرم هم اگر یاد سال اول لوون بیفتم قطعا اولین چیزی که میاد تو ذهنم تویی. ما با هم شروع کردیم به کشف لوون، از اون بخش یخچالی خنک کولرویت بگیر تا اون بهترین پیتزایی اودهورلی که تخم مرغ میشکوند رو پیتزاش. و بخش زیادی از تنهایی و غم و غربت سال اول رو با هم تحمل کردیم و شکوندیم. و البته سال دوم هم همینطور، که شرایط بهتر شده بود تا قبل از کرونا. هزار و یک خاطره هست تو ذهنم از این دوران، داستانایی که از ماجرای حساب بانکی باز کردن شروع میشه و بعد از طی دو سال میرسه به اون جمله "خانوم، شما" که مسئول تونل تست تب کرونا گفت تو تلاش دوم برای خروج از بلژیک. و قشنگ خودمون میدونیم تو این فاصله چقدر داستان، بخصوص از نوع آفتابه‌مآب وجود داره. دمت خیلی گرم رئیس، بابت همه این داستانها و خاطرات مشترک.
}
\end{flushright}

Hao, my friend from Japan, our collaborative work was one of the most enjoyable and straightforward experiences I’ve had in my PhD. I was impressed with your initiative in starting our collaboration, and it made me proud to be a part of it. I learned a lot from you, and I have no doubt that I'll have more questions for you in the future. I'm grateful for everything you did for me, including your assistance with writing part of my thesis. It's unfortunate that we never had the opportunity to meet in person, but I'm looking forward to it happening soon. I'm excited to see what other interesting projects your creative mind will come up with, and I hope that we can continue to collaborate in the future.

\begin{flushright}
\fontspec{XB Zar}
\foreignlanguage{persian}
{
مجید، و چون دوتا مجید داریم میگم که مجید ناظمی، استاد، سید و راهنما، و بزرگ مرد عرصه بداهه‌گویی (شما خود اصل واژه رو حدس بزن) در عین فوق آدم‌حسابی بودن، که دقیقا بخاطر همین خصیصه هیچوقت ازت سیر نمیشدم وقتهایی که میدیدمت. دوره‌ای که زیاد میدیدیم همو اگرچه خیلی کوتاه بود، و شاید از نظر زندگی شخصیم هم بهترین دوره این چند سالی که خارج از کشور گذروندم نبود، ولی بازم چرت نگفتم اگر بگم دوره خوبی بود چون تو رو زیاد (یا حداقل خیلی بیشتر از سالهایی که بعدش اومد) ملاقات میکردم. شما خود بخوان حدیث مفصل که چقدر عشق میکنم باهات. امیدوارم بازم زندگی فرصتی رو بده بهم که بتونم بیشتر با تو و صدالبته سمیه وقت بگذرونم. دمت گرم سلطان.
}
\end{flushright}

Marc, your friendliness and kindness made me feel welcomed from the moment I joined our research group. I will always cherish the memories of our conversations in the cozy GIGA center during my first year, discussing everything from life to politics to science and technology. Your unique personality and perspective on the world never failed to captivate me, and I only wish we could have spent more time together. It's a shame that we were separated by different locations, but still, I am so grateful to you for being such a wonderful friend and colleague.

Tim, the biology wizard, and the supportive man of our research group, I was lucky enough to work with you during my PhD. Your hard work and dedication to scientific discussions were always admirable, even though I couldn't always wrap my head around what you were doing. Despite being from different worlds, our collaboration was straightforward, and I think we should team up to write a guide on dealing with loom files, with a title like ``loom files for dummies''. It's the ultimate challenge and we're just the people to solve it!

\begin{flushright}
\fontspec{XB Zar}
\foreignlanguage{persian}
{
علی، دکتر پیر کوری ما. پسر فوق پرتلاش، که دقیقا بخاطر همین خصیصه بسیار ستایشت میکنم. خیلی خوشحالم که میشناسمت علی، و خیلی جاها مثالت زدم، به عنوان کسی که با تلاش و پشتکار زیاد مسیر زیادی رو طی کرده و به مدارج بالا رسیده. و خب شاید خودت اینو ندونی، ولی من چیزهای زیادی ازت یاد گرفتم که اتفاقا توی همین دوره دکترا هم خیلی بدردم خورد. پاندمی ما رو از هم دور کرد، ولی قبلش حتی بیرون دانشگاه هم دوران خیلی شیرینی رو با هم داشتیم. از اولین باری که با هم رفتیم و اوتاکوس رو بهم معرفی کردی بگیر بیا جلو همینطور. و دوتا کنفرانسی که با هم شرکت کردیم هم خیلی به من چسبید و خوش گذشت، مخصوصا توی پورتو، و فکر نمی‌کنم اون حد از خنده (یا کنترل خنده) تکرار بشه تو کنفرانس‌های دیگه‌ای برای من در آینده. خلاصه که دمت خیلی گرم دکتر.
}
\end{flushright}

Maria, I am so grateful to have met you during the early months of my PhD. Your kindness and support were immediately apparent, and I knew that this journey would be made easier with you in my corner. Your contribution to my research was invaluable, and I will always be thankful for your help. While it was sad that we couldn't continue working together, I'm thrilled that our friendship has grown even stronger. Your unwavering support has meant so much to me, and I appreciate them all especially the long travels you were always doing to attend our gigs. I can't wait to spend more time with you in the future!

Diego, my 5-string man and the greatest bassist I've ever met! I am thrilled that I had the opportunity to get to know you, particularly during my first year abroad. Our conversations were a constant source of inspiration for me. You have an extensive knowledge of bands and musicians, which made me realize how limited my music knowledge is in terms of diversity. The moments we shared together, cooking our favorite dishes, analyzing our favorite songs, and enjoying shots of delicious Colombian coffee, are memories that will last a lifetime.

Chola, Mr. Scientist, and the master of additive manufacturing. From the moment we met at the newcomers reception, I was impressed by your warm and welcoming personality. Your ever-present smile lit up every interaction we had, making them all even more enjoyable. We quickly discovered a shared passion for education and had plans to work together on many projects. Sadly, the world changed and we were unable to make those plans a reality. Still, I'm grateful for the wonderful discussions and fun moments you made. Thank you so much man.

Tiago, you're one of the coolest people I've ever met, and I'm grateful for the brief time we were officemates. In just 7 months, you showed me how to be cool without trying too hard. You've gone on to achieve so much success, and I'm proud of you for that. Your approach to life inspired me to strive for happiness and personal growth. And your influence even extends to the way I'm writing this thesis acknowledgment. I've heard from others that the BMe environment that we experienced was completely different before you, and I can only say thank you for the positive impact you had on all of us.

\begin{flushright}
\fontspec{XB Zar}
\foreignlanguage{persian}
{
متین، مرد سخت‌کوش روزهای سخت، آدم فوق باجنبه که هیچوقت از سیل شوخی‌های مسخره من ناراحت نشد، و یکی از پایه‌ترین آدم‌های دنیا مخصوصا از ساعت شش عصر به بعد. من هیچوقت لذتی که پشت این ترکیب "ساعت شش" هست رو فراموش نمیکنم پسر، روزهایی که لوون بودی جزو بهترین و پرخاطره‌ترین دوران من توی لوونه، روزهایی که به بدترین نحو ممکن تموم شد با شروع شدن پاندمی، ولی حتی همون خداحافظی رو هم خاطره کردیم تو اوج دوران ترس و شیوع و دلهره. و ارتباطی که بعدش قطع نشد و با اشتیاق ادامه پیدا کرد، و بعدش روزهای خوبی که اومد بالاخره برات. خیلی خوشحالم که با هم آشنا شدیم و اینهمه خاطره با هم داریم. کیپ این تاچ پسر، کیپ این تاچ.
}
\end{flushright}

Aurelie and Rick, my dear colleagues from Maastricht, working with you was an incredible learning experience, and I'm so glad that we had the opportunity to collaborate. Rick I have always appreciated our conversations about the development of the model and the underlying science, even though I must admit my ignorance when it comes to the biological aspects. Aurelie, your remarkable scientific achievements and your strong work ethic make you a true role model for me. I have immense respect for your dedication to academia and very happy to know you.

Helmhotz people, Sviatlana, Di, and Cheng, this PhD work owe you a lot, and I’m pretty sure you know your role in shaping it. You helped me to align my research with the practical needs of the applied chemistry field, and your expertise and insights have been invaluable. I look forward to continuing to learn from you in the future and to applying the knowledge and skills I gained from working with you to my future endeavors. Thank you so much for everything.

BMe people, Andrea, Markos, Rebeca, Klaas, Tommy, Hannelore, Harry, Jos, Kimberly, Amelie, Stijn, George, Stefan, Apeksha,  Jilmen, Laura, Lauranne, Maite, Paulien, Heleen, Hadi, Julie, Nele, Quentin, and Marie-Mo, for four years, BMe has been my scientific home, and I always felt missing it whenever I spent time anywhere else. Undoubtedly, the working environment played a crucial role in this sentiment, and I attribute it entirely to the wonderful personalities of my colleagues. I am grateful to have had the opportunity to work with such amazing individuals. Thank you for being such great colleagues. I should also specially thank Jana and Kristel, the endless source of energy with always-smiling faces. You were always there when I needed you, and this for sure means a lot to me. Thank you for being such an integral part of our team.

Bandmates Nard, Rocco, and Parnian, I feel fortunate to have met you through our shared passion for music in "Me, Moon and You". I am grateful for the evenings we spent rehearsing for gigs and the memorable times we had together after the shows. Although our time together was relatively short, I hope that we have the opportunity to create music together once again in the future.

UvA people, Gabor, Christian, Yue, Dongwei, Niksa, David, Vittorio, and Eric, thank you all wonderful 3 months I spent at the University of Amsterdam. Being a part of your research group and experiencing your working environment was truly an enriching experience for me. Gabor, your expertise and guidance helped me learn and gain new knowledge on the state-of-the-art HPC, and I am forever grateful for that. The discussions we had in your office, the Friday's pub sessions, and the joyful lunch gatherings were always insightful and inspiring. I cannot thank you enough for the warm welcome and the incredible support you showed me during my stay.

Biomech people, Bingbing, Edo, Gabriele, Ayse, Niki, Raphaelle, Mahtab, Bernard, Sourav, Sebastian, Azadeh, Ahmad, Morgan, Mervenaz, Claire, Fabio, Luiz, Alessio, Lisanne, Liesbeth, Yun, Reza, and Lillian. I always enjoyed all the scientific and non-scientific events we spent together, along all the unofficial chats and nice talks we always had whenever I had the chance to meet you. I will never forget the wonderful team days (although I could only attend two of them) and amazing weekend and conference trips we went to together. Those memories will stay with me forever.

During my PhD, I spend a significant amount of my time in Eindhoven, and I am incredibly grateful for the wonderful people who made my stay in this city so enjoyable. Although most of my friends and hobbies were connected to Leuven, the presence of certain individuals in Eindhoven made my life colorful and fun during this second half of it in a completely different atmosphere and country. For sure, there are quite a few people to name here, but I would like to express my gratitude to Sadra, Afsaneh, Amoo Radi, and Mahtab for all the joyful moments we shared together. Living abroad is unpredictable, but I hope you all remain our life-long friends, no matter what the future holds.

And talking about Eindhoven, there is another appreciation I should make here regarding the support I received from my new colleagues/friends to overcome the challenges I had for wrapping up the last stages of my PhD. Toni, Adrian, Maxime, Marit, Senan, Rik, Rens, Remy, Jesse, Irene, Mert, Winnie, Jeroen, Emre, Inma, Simona, Aylin, Nadia, Menno, and Victor, thank you so much for providing me with such an amazing and welcoming environment at TU/e. I could not have overcome the challenges without your warm support. Your positive attitude made my transition to the next chapter of my life much smoother. 

\begin{flushright}
\fontspec{XB Zar}
\foreignlanguage{persian}
{
و میرسیم به یک جنس دیگه از حضور و اثر، حضوری که گرچه جنسش از جنس حضور فیزیکی نیست، ولی اثرش یقینا چیزی کمتر نداشته برام. دوستانی که از راه دور و نزدیک هوام رو داشتن، هوا داشتنی درست مثل همه سالهایی که پشت سر گذاشتیم قبل از اینکه این مسیر جدید، اینجا، شروع بشه. محمدرضا، رفیق هفده ساله، آقابهرامی با همون تلفظ همیشگی، یار گرمابه و گلستان، که شاید خودت ندونی ولی تاثیر بینهایت بزرگی داشتی توی زندگی من و البته این تزی که داره نوشته میشه. حسین خان، سلطان تحلیل و موشکافی مسائل، که خود این بحث‌ها یه نقطه اتصال مهم بود به اون نسخه‌ای از خودم که خیلی بیشتر از خود فعلیم شاید دوسش دارم، و هزاران کمک دیگه‌ای که همیشه حرفات بهم کرده. عاطفه، و صد البته عاطفه، با حضور بسیار تاثیرگذار، حمایت بی‌دریغ، و تاریخچه‌ای از بینهایت اشارات نظر که بینمون هست و خواهد بود. قاسم و سارا، رفقای جان، که باعث میشدین این حس رو داشته باشم که حداقل دو نفر آدم دارم توی قاره به این درندشتی که بشه بهشون پناه برد. علیرضا، دکترجون خودم، آدم حسابی آدم حسابیا، که رفتی نشستی یه جایی که حس برادر بهت دارم. و سهیل، رفیق تازه‌تر از راه رسیده، که هیچوقت از خندیدن باهات سیر نشدم. دمتون گرم رفقا، تقریبا مطمئنم که خودتون میدونین که چقدر برام عزیزین و در حکم گنجینه‌ای پر ارزش. خوشحالم که دارمتون.
}
\end{flushright}

\begin{flushright}
\fontspec{XB Zar}
\foreignlanguage{persian}
{
برای از خانواده نوشتن یک پاراگراف کمه، خیلیم کمه. یه پاراگراف رو تا بخوای شروع کنی تموم شده. از کجاش بگم اصلا؟ از کجا شروع کنم؟ از دلتنگی بگم یا از این بگم که بهترین خانواده دنیا رو داشتم و دارم؟ از بهترین پدر مادر دنیا بنویسم یا از همراهترین برادر یا از پدرترین عموی تاریخ؟ هر چی دارم رو از شما دارم، به معنای واقعی کلمه. ذره ذره خصلت‌های اخلاقی و حرفه‌ای، توی همشون پررنگ‌ترین ردپاها از شما هست. سالهای سختی رو با هم پشت سر گذاشتیم، ده دوازده سال خیلی سخت رو، سالهایی پر از فشار و استرس، که صدالبته پشتمون شدید به پشت هم گرم بود، و وقتی تازه داشت این فشاره یکم کم میشد من گذاشتم رفتم. و این شاید سختترین تصمیم عمرم بوده و خواهد بود. سالهای دور از خانه، برای منی که دور بودن از خونه رو بلد نبودم، و این فکر شبانه روزی توی تمام این چهار سال و خورده‌ای که آیا کار درستی کردم یا نه. ولی قطعا به یه چیز هیچوقت شک نمیکنم، اینکه اگر حمایت شما از من و این تصمیم نبود خیلی خیلی زودتر از اینها کم میووردم. اینکه از فرسنگ‌ها دورتر هم میشه دلگرم بود به حضور چند نفر، حسی بود که تجربه کردنش ورای همه تجربه‌های قبلیم بود و خیلی فرق داشت با انتظارم از حس‌هایی که فکر میکردم قراره تو زندگی سراغم بیان. بابا مجید، مامان عفت، عمو مهرداد، مرتضا، مطهره عزیز (که چه خوش اومدی به زندگیمون)، و علیرضا کوچولوی ما (که الان شده بیست و خورده‌ای سالت ولی همون علی کوچولوی مایی هنوز)، خیلی خیلی خیلی خوشحالم که شما خانواده من هستین، و در اوج افتخارم از داشتن خانواده‌ای مثل شما، و تا ابد مدیونتون.
}
\end{flushright}

\begin{flushright}
\fontspec{XB Zar}
\foreignlanguage{persian}
{
مرضیه، مرضیه من، خانوم هاشمی، از همون اول هم میدونستم که تشکر و قدردانی از تو احتمالا سختترین قسمت از کل این نوشته خواهد بود. اینکه اصلا باید از کجا شروع کنم به قدردانی، خودش داستانیه. و وقتی میگم قدردانی، به معنی واقعی کلمه یعنی قدر دانستن، یعنی دیدن ارزش واقعی حضور، و نقش بسیار پررنگ در تمام اتفاقی که افتاده برای این دوره از زندگیم. فکر کنم همین بس که بگم حضور تو باعث شد که اصلا این پروژه اپلای و دکترا شکل بگیره، با ساختن دوباره من از چیزی که به کلی از دست رفته بود و در گذر زمان ویران شده بود. از اون ویرانه تو شروع کردی به ساختن، و ساختی و ساختی، تا رسید به اینجا، و فقط شاید خودمون بدونیم که چقدر ساختیم، که بیشترش رو تو ساختی. و شاید من اگر بخوام تو رو فقط و فقط تو یه کلمه توصیف کنم، بگم سازندگی. ممنوتم بابت همه این ساختن‌ها، بابت همه رنگی که بخشیدی به زندگی من، و بابت حضور بسیار قویت تو تک تک لحظات سخت این چندین سال اخیر. ما هیچوقت نمیدونیم آینده قراره چی برامون بیاره، ولی همیشه از این مطمئن خواهم بود که با هم میریم جلو و مثل همین سالهایی که پشت سر گذاشتیم، همین سالهایی که کیف کردیم و پشت سر گذاشتیم، میسازیم و میریم جلو. خیلی دوس‌ت دارم. مرسی که اومدی به زندگی من، و مرسی که هستی.
}
\end{flushright}

And in the end of this long acknowledgment (about which I have no regret of course!), I would like to express my gratitude towards those who have had a significant impact on shaping the person I am today in my adulthood. While there are numerous names to mention, I would like to highlight Richard Stallman and Jadi Mirmirani as two of the most influential individuals who have inspired me for over a decade. Lastly, I would like to extend my hopes for a brighter future for my home country, Iran, which has suffered from many challenges over the past century, and for the world to experience the spreading of peace amidst the current dark times.

Mojtaba (aka Moji)\\
Leuven - Eindhoven\\
June 2023




%%%%%%%%%%%%%%%%%%%%%%%%%%%%%%%%%%%%%%%%%%%%%%%%%%
% Keep the following \cleardoublepage at the end of this file, 
% otherwise \includeonly includes empty pages.
\cleardoublepage

% vim: tw=70 nocindent expandtab foldmethod=marker foldmarker={{{}{,}{}}}
