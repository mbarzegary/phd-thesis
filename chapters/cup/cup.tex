% !TeX root = ../../thesis.tex
\chapter{Applications: Acetabular cup}\label{ch:cup}

\begin{shaded}
This chapter is based on a manuscript prepared to be submitted:\\
M. Barzegari, F. Perez-Boerema, G. Zavodszky, and L. Geris, ``High-performance computer simulation of biodegradation of optimized personalized implants; a case study of a patient-specific porous acetabular implant.''
\end{shaded}

\section{Introduction}


3D-printed orthopedic implants have been gaining  popularity in recent years due to the control this manufacturing technique gives the designer over the different design aspects of the implant \cite{Kumar2021,Yadav2020}. This technique allows manufacturing implants with material properties similar to bone, giving the implant designer the opportunity to address complications experienced after various surgeries, an example of which is the aseptic loosening of the implant after total hip arthroplasty (THA). The rate and quality of bone regeneration after implantation of orthopaedic implants depends greatly on the achieved mechanical stability. To restore proper function after implant loosening, the implant needs to be replaced. During these revision surgeries, some extra bone is removed along with the implant, further increasing the already present defects, and making it harder to achieve proper mechanical stability with the revision implant \cite{Luthringer2014}. A possible way to limit the increasing loss of bone is the use of biodegradable orthopedic implants that optimize long-term implant stability, in which part of the implant, like the acetabular cup shown in Fig. \ref{fig:cup_implant}, is made from biodegradable metals. This means that the biodegradable part of the implant will disappear and be replaced by newly formed bone during the implant's lifetime. Taking advantage of these implants needs to both optimize the implant such that stress shielding is minimized, and tune the implant degradation rate such that newly formed bone is able to replace the degrading metal in order to maintain a proper bone-implant contact. The hope is that such (partly) degradable implants will lead to a reduction in the size of the bone defects over time, making possible future revisions less likely and less complex.

\begin{figure}[h]
\centering
\medskip
\includegraphics[width=\textwidth]{implant.jpg}
\caption[Application of the acetabular implant]{Demonstration of the application of a acetabular implant which has a porous structure on its back made from biodegradable metals, a) the large bone defects in the hip bone, b) the implant used for fixing the problem in the total hip arthroplasty surgery, c) the biodegradable porous structure on the back of the implant. } \label{fig:cup_implant}
\end{figure}

In this study, we focused on improving the long-term implant stability of patient-specific acetabular implants for large bone defects and tuning of their biodegradable behavior. To improve long-term implant stability, we implemented a topology optimization approach, in which a patient-specific finite element model of the hip joint with and without implant was derived from CT-scans to evaluate the performance of the designs during the optimization routine. The implant shape was generated using a lattice infilled structure based on the results of the performed topology optimization.

Developing a quantitative mathematical model of the degradation process is a proper solution for tuning the biodegradation and material release rate, allowing researchers to study the biodegradation behavior of any desired implant \textit{in silico} (in the computer) prior to conducting any \textit{in vitro} or \textit{in vivo} experiments. Developed mathematical models can be simulated using efficient numerical schemes such as the finite element method. The primary challenge here will be achieving a high numerical accuracy at the interface between the implant and surrounding tissue in the body as the interface plays an important role in the degradation phenomena. Increased accuracy means increased computational cost and resources (especially time), but high-performance computing (HPC) techniques can be used to overcome this challenge.



\section{Materials and Methods}

\subsection{Topology optimization model}

For building the lattice structure, the in-house developed open-source tool ASLI \cite{Perez-Boerema2022} was used to create a triply periodic minimal surface based infill for the acetabular implant based on the output of a patient-specific topology optimization.

\subsection{Biodegradation model}




In this study, we developed a quantitative mathematical model to predict the biodegradation of magnesium-based implants. Magnesium (Mg) has been selected to start with due to its acceptable mechanical properties, biocompatibility, and contribution in osteoinductivity \cite{Agarwal2016}. The developed model captures the release of Mg ions, the formation of a protective film,  the dissolution of this film due to the presence of various ions in the surrounding environment, and change of pH. This was accomplished by deriving a system of nonlinear time-dependent reaction-diffusion partial differential equations (PDEs) from the underlying oxidation-reduction reactions occurring during the biodegradation process of metallic materials in simulated body fluid. The level-set formalism was employed to track the movement of the biodegradation front between the implant and its surroundings. The equations were solved implicitly using the finite element method for spatial terms (with a 1st order Lagrange polynomial as the shape function) and backward-Euler finite difference method for temporal terms on an Eulerian mesh, implemented in the open-source PDE solver FreeFEM\cite{Hecht2012}. The details of this model can be found in our previous contribution \cite{Barzegari2021}. Two separate simulations were performed using two different diffusion coefficients as calibrated in our previous work to model the degradation behavior in non-buffered and buffered solutions.

In order to build the computational model, the resulting surface mesh of the topology optimization routine, consisting of \num{5347924} faces, was converted to a volume mesh and embedded in a cubic container that was to act as the environment during the biodegradation simulations. The conversion of the surface mesh was done using GMSH \cite{Geuzaine2009}, and the embedding process was carried out using the internal mesh engine of FreeFEM, called BAMG. The mesh was refined on the implant-environment interface to increase the numerical accuracy of the interface tracking model, leading to a final mesh with \num{45870053} elements. The mesh refinement process was handled using Mmg \cite{Dapogny2014} (and ParMmg \cite{balarac:hal-03344779}). Fig. \ref{fig:cup_model} shows the embedded cup model and a cross section of the final generated mesh, which was refined on the metal-solution interface.



\begin{figure}[h]
\centering
\medskip
\includegraphics[width=\textwidth]{model.jpg}
\caption[Computational biodegradation model for the porous acetabular implant]{a) The computational biodegradation model for the porous acetabular implant: a)the optimized acetabular implant embedded inside a cubic container, b) a cross section of the generated mesh with the surface of the implant visualized as the light gray surface.} \label{fig:cup_model}
\end{figure}

For dealing with a problem of this size and making the model capable of being scaled in large-scale computing systems, the model was implemented making use of the high-performance computing (HPC) techniques available in the FreeFEM language v4.10 and PETSc toolkit v3.16.1 \cite{petsc}. In this implementation, METIS and ParMETIS graph partitioners \cite{METIS1998} were used to decompose the mesh into various partitions, and then the partitioned mesh was distributed among available computing resources using the HPDDM preconditioner \cite{Jolivet2013}. Fig. \ref{fig:cup_decomposition} shows the decomposition of the computational mesh. Moreover, HYPRE BoomerAMG \cite{Falgout2002} preconditioner and GMRES iterative solver \cite{Saad1986} were used to solve the linear systems of equations obtained from applying the finite element discretization on the model. More details of the implementation can be found in \cite{Barzegari2022}.

\begin{figure}[h]
\centering
\medskip
\includegraphics[width=\textwidth]{decomposition.jpg}
\caption[Mesh decomposition of the acetabular implant model]{Mesh decomposition of the computational biodegradation model to be distributed to available computing nodes, a) top view, b) perspective side view. } \label{fig:cup_decomposition}
\end{figure}

The simulation was carried out using 2,000 CPU cores with 16.5 TB of available memory on Dutch national supercomputer, Snellius. The simulation results, comprising of \num{95000} files with a total size of 148 GB, were visualized using a parallel client-server remote rendering approach in ParaView server v5.9.1 running on 128 CPU cores on the ARCHER2 supercomputer.

In order to obtain the scaling behavior of the model in an HPC environment, strong and weak scaling tests were performed in the Snellius supercomputer. For the weak scaling test, in addition to the mesh with \num{45870053} elements, two more mesh files consisting of less elements were generated using the aforementioned procedure for embedding and refining the mesh. These models had \num{15989521} and \num{29035491}, respectively, to which the number of employed CPU cores were adjusted accordingly. The strong scaling test was carried out for all the three model sizes by varying the number of employed CPU cores from 60 to \num{9000}.


\section{Results and Discussion}

\subsection{Topology optimization of the acetabular implant}

The generated lattice structure of the implant, infilled by skeletal-gyroid unit cell type, is depicted in Fig. \ref{fig:cup_optimization}. The volume fraction is varied during the topology optimization process to match the stiffness distribution required to maximize the tissue growth.

\begin{figure}[h]
\centering
\medskip
\includegraphics[width=\textwidth]{optimization.jpg}
\caption[Infilled acetabular implant]{Acetabular implant infilled by varying volume fraction to match a desired stiffness distribution.} \label{fig:cup_optimization}
\end{figure}

\subsection{Biodegradation of the infilled implant}

Fig. \ref{fig:cup_degradation_rate} shows the mass loss during the biodegradation for simulations performed in buffered (low degradation rate) and non-buffered (high degradation) solutions. The mass loss is one of commonly-used indicators for the degradation rate, demonstrating that the biodegradation rate is much higher in the non-buffered solution. Loss of material over time was found to be in line with the values obtained in our previous study \cite{Barzegari2021}, showing that scaling the model in an HPC environment does not affect the quantitative predictions made by the model. The developed mechanistic model of the biodegradation process includes a level-set equation correlating the loss of material to the velocity at which the implant interface shrinks. A good agreement of results on large scale (a model with $\sim$45M elements) shows that the interface tracking formulation behaves efficient even in problems with a high level of details. This verifies the performance of the developed biodegradation model, which has never been tested in such high resolutions.

\begin{figure}[h]
\centering
\medskip
\includegraphics[width=0.9\textwidth]{degradation_rate.png}
\caption[Biodegradation rate for the acetabular implant]{Rate of mass loss during the biodegradation of the porous acetabular implant in saline and buffered solutions.} \label{fig:cup_degradation_rate}
\end{figure}

From a qualitative point of view, visualizing the biodegradation results, depicted in Fig. \ref{fig:cup_degradation_visual} over time, shows that the acetabular implant degrades faster in the regions with higher porosity, i.e. the region with more exposed surface area to the environment. These regions have lower stiffness, which resulted in high porosity by being filled with TPMS unit cells having low volume fraction. This correlation can be discussed from the pathological perspective as well, meaning that the low stiffness parts get replaced faster by the newly formed bone during the bone healing process. It is the expected behavior of the porous implant and the aim of such a patient-specific design.

\begin{figure}[h]
\centering
\medskip
\includegraphics[width=\textwidth]{degradation_visual.jpg}
\caption[Visualization of the change of morphology of the acetabular implant]{Visualization of the change of morphology of the acetabular implant over time.} \label{fig:cup_degradation_visual}
\end{figure}

Fig. \ref{fig:cup_degradation_visual_close} demonstrates a similar visualization but on a zoomed view to the surface of the implant being plotted along with a cross section of the medium showing the concentration of released Mg ions. The movement of the corrosion interface is formulated based on the release of material ions, and as a result, the material loss rate is higher in regions with higher ions concentration. The  concentration is directly correlated to exposed surface area, meaning that higher surface to volume ratio results in higher material release. This explains why the regions with higher porosity degrades faster in the current model, which is a realistic behavior captured by the developed mathematical model. 

\begin{figure}[h]
\centering
\medskip
\includegraphics[width=\textwidth]{degradation_visual_close.jpg}
\caption[Visualization of the change of morphology of the acetabular implant]{A closer look at the visualization of the change of morphology of the acetabular implant over time along with a cross section of the medium showing the concentration of released Mg ions.} \label{fig:cup_degradation_visual_close}
\end{figure}

\subsection{Scaling tests on the computational models}

The results of the strong scaling tests are plotted in Fig. \ref{fig:cup_strong_scaling}, which shows the solution time of individual comprising equations of the biodegradation model in a single time step versus the varying number of CPU cores. The run time of a single time step was measured for three model sizes of $\sim$16M (small), $\sim$29M (medium), and $\sim$45M (big) elements. As can be seen in the strong scaling results, different PDEs show different scaling behavior, the most distinct of which belongs to the OH equation used to calculate the change of pH in the surrounding environment. The different scaling behavior routed in the boundary conditions used for each equation, and since penalization technique is used to implement boundary conditions, they impact the formation of the linear system of equations and appear in the scaling results as well.

The strong scaling test shows that the optimal size for the distributed computing environment is \num{5000} CPU cores, above which no significant improvement in the execution time was observed in all the three tested problem sizes. However, the scaling behavior of models slightly differs depending on the size. As expected, in problems with smaller size, the costs associated with inter nodes communication become effective faster in comparison to bigger models, and as a result, increasing the number of nodes becomes useless in earlier stages in these model sizes. This fact can be seen in the cumulative strong scaling plots (the plot in the bottom right), where the graph for the small model tends to reach a vertical line earlier in comparison to the graph for the big model.

\begin{figure}[h]
\centering
\medskip
\includegraphics[width=\textwidth]{strong_scaling_vertical.png}
\caption[Strong scaling of individual components of the biodegradation model]{Strong scaling of the computational model, performed using the small, medium, and large mesh for the solution of individual comprising equations of the biodegradation model, in which the execution time is plotted versus the varying CPU cores in the logarithmic scale.} \label{fig:cup_strong_scaling}
\end{figure}

Fig. \ref{fig:cup_weak_strong} shows the results of the strong and weak scaling tests side by side. As can be seen in the weak scaling result, the model shows an ideal scaling behavior in environments with less than \num{1000} cores. With the number of CPU cores more than this, the system shows non-ideal behavior, as can be observed in strong scaling results, in which the addition of more CPU cores does not cause a linear drop in the execution time (wall time). 

The weak and strong scaling results clearly indicate that the code cam be improved to increase the performance of the model, considering the fact that even taking 10K CPU cores is not a big thing to tackle in the current state of HPC and typical scales used for biomedical related simulations and computational modeling works. The first part to improve seems to be the assembling of the linear system, in which the schemes used for numerical integration, order of elements used for various involved physics, and the mthod for applying boundary conditions play an important tole. Secondly, the configuration of employed preconditioners and iterative solvers can be modified for individual components according to the numerical computing requirements, which can affect their scaling behavior (Fig. \ref{fig:cup_strong_scaling}). However, making a more concrete conclusion on the performance bottlenecks of the developed model needs further analysis of the results for the parallel speedup, similar to the analysis performed in our previous work \cite{Barzegari2022}. Furthermore, it should be taken into account that such performance analysis has an inherent simplification of more complex technical aspects affecting the simulation execution time, such as load balancing, network communication, and per node limitations in memory access, which may differ in each HPC environment due to the configurations made by the maintainers.


\begin{figure}[h]
\centering
\medskip
\includegraphics[width=\textwidth]{weak_strong.png}
\caption[Weak and strong scaling of of the acetabular implant model]{Weak and strong scaling of of the computational biodegradation model of the acetabular implant, plotted for the total time needed to solve all the equations in a single time step.} \label{fig:cup_weak_strong}
\end{figure}


\section{Conclusions}

In this work, taking advantage of HPC techniques to simulate a large-scale 3D model led to a computational model capable of predicting the biodegradation behavior of an acetabular implant in high resolution. Results demonstrate the potential of the model to act as a tool for assessing and tuning the biodegradation properties of orthopedic  implants regardless of shape or complexity.

\section{Acknowledgments}

This research is financially supported by the Prosperos project, funded by the Interreg VA Flanders – The Netherlands program, CCI grant no. 2014TC16RFCB046 and by the Fund for Scientific Research Flanders (FWO), grant G085018N. The results were obtained during a research visit to the Computational Science Lab at the University of Amsterdam, funded and supported by FWO long stay grant V408622N. The computational resources and services used in this work were provided by SURF (www.surf.nl) using the Dutch National Supercomputer Snellius. The visualization works used the ARCHER2 UK National Supercomputing Service (www.archer2.ac.uk).


%\begin{subappendices}

%\section{Challenges in scaling the computational model to thousands of CPU cores}

%Building with different MPI implementations
%Inter-node communication
%Running parallel FF
%Mesh generation (parallel, sequential)
%Converting surface mesh to volume
%Scaling issues
%Memory issues
%Node types
%Visualization, GPU nodes
%Partitioning (METIS, ParMETIS)
%Storage
%Solving NS equation

%\end{subappendices}


%%%%%%%%%%%%%%%%%%%%%%%%%%%%%%%%%%%%%%%%%%%%%%%%%%
% Keep the following \cleardoublepage at the end of this file,
% otherwise \includeonly includes empty pages.
\cleardoublepage
