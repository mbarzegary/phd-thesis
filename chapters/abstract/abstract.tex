% !TeX root = ../../thesis.tex
\chapter{Abstract}                                 \label{ch:abstract}

Degradable and solvable metallic materials are gaining popularity in a wide variety of applications. In the biomedical field and for applications in medicine, biocompatibility, biodegradability, and positive contribution to the metabolism of the human body are the critical properties that nominate a metal as an applicable option. By taking this into account, magnesium (Mg), iron (Fe), and zinc (Zn) are usually considered as biodegradable metallic materials. Due to the great mechanical properties of metals in comparison to ceramics and polymers, metallic biomaterials are appropriate candidates for load-bearing conditions in various bone healing and cardiovascular applications. An important advantage of using biodegradable materials for these applications, which usually involve temporary supporting functions, is avoiding complications caused by permanent materials, such as eventual stress-shielding, contribution to the weakening of the bone, needing revision surgery to remove, and conflicts with diagnostic imaging. In addition to this, especially for bone healing applications, the released metallic ions during the degradation process contribute to the metabolism of the underlying biological process. For example, Mg is one of the most abundant ions found in the bone, and Mg cations have a beneficial impact on the metabolism of enzymes in the bone regeneration process. Similarly, Fe plays a key role in oxygen transport in the body, and Zn positively influences the physiological functions of bone healing and the formation of different transcriptional factors.

Despite the advantages of using biodegradable metals in implant design, their fast degradation and uncontrolled release remain a challenge in practical applications. Beside experimental approaches to investigate the properties of biodegradable metallic implants and scaffolds, computational modeling of the biodegradation process and behavior can act as an efficient tool to design the next generation of medical devices and implants. A validated computational model of the degradation process can facilitate tuning of biodegradation properties. In this study, we have developed a mathematical and computational model to predict the biodegradation behavior of biodegradable metallic biomaterials. Our developed model captures the release of metallic ions, changes in pH, the formation of a protective film, the effect of different ions in the environment, and the effect of perfusion of the surrounding fluid, if applicable. This has been accomplished by deriving a system of time-dependent reaction-diffusion-convection partial differential equations from the underlying oxidation-reduction reactions and solving them using the finite element method. The level set formalism was employed to track the biodegradation interface between the biomaterial and its surroundings. The model was validated by comparing the predicted and experimentally obtained values of global and local pH changes in corrosion tests, for which a good agreement was observed.

Tracking the moving front at the diffusion interface requires high numerical accuracy of the diffusive state variables. Improving the accuracy requires a refined computational mesh, leading to a more computation-intensive simulation. To overcome this challenge and yield more interactable simulations, scalable parallelization techniques were implemented, making the model capable of being run on massively parallel systems to reduce the simulation time. Hereafter, the scaling behavior of the models were evaluated on hundreds to thousands of CPU cores in high-performance computing environments. 
Additionally, the core biodegradation model was coupled with fluid flow models to enable capturing the effect of hydrodynamics and perfusion conditions. Moreover, the model was employed in a couple of multiphysics use-cases as the biodegradation compartment to demonstrate the ability of the model to be integrated in other modeling workflows in biomedical engineering.

%%%%%%%%%%%%%%%%%%%%%%%%%%%%%%%%%%%%%%%%%%%%%%%%%%
% Keep the following \cleardoublepage at the end of this file, 
% otherwise \includeonly includes empty pages.
\cleardoublepage

% vim: tw=70 nocindent expandtab foldmethod=marker foldmarker={{{}{,}{}}}
