% !TeX root = ../../thesis.tex
\chapter{Tissue growth modeling}\label{ch:tissue}

\begin{shaded}
This chapter contains partial results from previously published content in \textit{Advanced Functional Materials}:\\
D. Van, B. Liang, S. Anania, M. Barzegari, B. Verlée, G. Nolens, J. Pirson, L. Geris, F. Lambert, ``3D-Printed Synthetic Hydroxyapatite Scaffold With In Silico Optimized Macrostructure Enhances Bone Formation In Vivo,'' \textit{Advanced Functional Materials}, vol. 32, p. 2105002, 2022.
\end{shaded}


\section{Introduction}

Numerical tracking of interface movement has been used for certain modeling applications in science and engineering for multi-material and multiphase problems such as solidification, melting, corrosion, and grain growth to name a few \cite{Sun2007}. The most popular Eulerian methodologies in this regard are the level-set \cite{Osher1988,Andrew2000,RonaldFedkiw2002}, volume-of-fluid (VOF) \cite{Rider1998}, and phase-field methods \cite{Boettinger2002,Bellemans2017}. 


The basic idea of the level-set method is to employ the Hamilton-Jacobi (HJ) algorithm for solving the general interface advection equation. The independent variable in the level-set method is a signed distance function, called the level-set function $\psi$ \cite{RonaldFedkiw2002}. The level-set function should be re-initialized as the interface evolves, which is the reason behind an inadvertent mass loss, one of the most prominent shortcomings of the level-set method. Although the VOF method is not vulnerable to the mass loss issue, calculating the interface curvature is difficult from the volume fraction \cite{Sun2007}. 

To overcome these challenges, diffuse interface methods \cite{Anderson1998} have gained attentions in recent decades, among which the phase-field method has shown potential for solving complex moving interface problems. Contrary to the level-est method, in diffuse interface methods, the interface is considered as a smooth transition between phases, which usually has a finite width. In the phase-field method, a non-conserved (or conserved) order parameter $\phi$ is defined such that $\phi=1$ in one bulk phase and $\phi=-1$ in the other. Then, the smooth transition between these two phases ($-1<\phi<1$) is marked as the interface. One of the advantages of the phase-field method is that the derived equation can be solved over the entire domain of desired without considering the location of the interface. Moreover, although the curvature and interface normal vectors are not formulated explicitly, the phase-field method is suitable for problems in which the evolution of the interface depends on the local curvature or a field acting normal to the interface \cite{Sun2007}. The phase-field method keeps a constant thickness for the smooth transition region normal to the interface, and as a result, no re-initialization as for the level-set method is needed.



The phase-field method has been already proved to be an efficient interface tracking technique for various problems in micro/meso scales such as solidification \cite{Karma1998,Boettinger2002}, microstructural evolution \cite{Chen2002}, grain growth \cite{Chen1994}, crack propagation \cite{Henry2004,Spatschek2011}, electromigration \cite{Bhate2000}, and extractive metallurgy \cite{Bellemans2017}. However, it has been used recently for dealing with problems described in macro level, such as corrosion \cite{Mai2016,Lin2019,Imanian2018,Lin2020,Ansari2018,Tsuyuki2018,Chadwick2018} and cell/tissue growth \cite{Jeong2017,Lee2019}. In this work, a phase-field model of the tissue growth process was developed to describe the  cell growth behavior on 3D surfaces as a moving-boundary problem. Additionally, a similar model was developed based on the level-set method to compare the performance and results of both interface tracking methods. Both models were implemented using the finite element method. 


\section{Deriving the model}

In this section, the derivation of the phase-field and level-set equations from the general advection equation is demonstrated. This shows the similarities and differences of these interface tracking techniques for moving boundary problems.

\subsection{General equation of interface motion}

The general interface advection equation for an Eulerian description of interface movement can be written as \cite{Sun2007}:

\begin{equation} \label{eq:general_advec}
\frac{\partial \phi}{\partial t}+\boldsymbol{u} \cdot \nabla \phi=0
\end{equation}
where $\phi$ is the phase field and $\boldsymbol{u}$ is the interface velocity. The velocity $\boldsymbol{u}$ can be divided into normal ($u_{\mathrm{n}}$) and external velocity components ($\boldsymbol{u}_{\mathrm{e}}$):
\begin{equation} \label{eq:two_comp_advect}
\boldsymbol{u}=u_{\mathrm{n}} \boldsymbol{n}+\boldsymbol{u}_{\mathrm{e}}
\end{equation}
in which $\boldsymbol{n}=\nabla \phi /|\nabla \phi|$ is the unit vector normal to the interface. So, Eq. \ref{eq:general_advec} can be rewritten as:
\begin{equation}
\frac{\partial \phi}{\partial t}+u_{\mathrm{n}}|\nabla \phi|+\boldsymbol{u}_{\mathrm{e}} \cdot \nabla \phi=0
\end{equation}

The normal velocity can be decomposed into more components to take into account the effect of interface curvature ($\kappa$) such that the terms are independent and proportional to the curvature, respectively:
\begin{equation}
u_{\mathrm{n}} = a - b \kappa
\end{equation}
where the coefficients $a$ and $b$ have units of $\mathrm{m}/\mathrm{s}$ and $\mathrm{m}^2/\mathrm{s}$. Substituting this into Eq. \ref{eq:two_comp_advect}  yields to the final form of the interface motion equation:
\begin{equation} \label{eq:advect_kappa}
\frac{\partial \phi}{\partial t}+a|\nabla \phi|+\boldsymbol{u}_{\mathrm{e}} \cdot \nabla \phi=b \kappa|\nabla \phi|
\end{equation}


\subsection{Phase-field formulation}

To further proceed with the phase-field formulation, a proper kernel should be selected for the phase field variable, which can be done based on Beckermann et al. \cite{Beckermann1999}:
\begin{equation} \label{eq:kernel}
\phi=-\tanh \left(\frac{n}{\sqrt{2} W}\right)
\end{equation}
in which $W$ is the thickness of the transition profile ($\phi$ varies from $-0.9$ to $+0.9$ in a narrow layer with the width of $3 \sqrt{2} W$), and $n$ is the coordinate normal to the interface. The curvature can be written as a function of the phase field variable:
\begin{equation}  \label{eq:curvature}
\kappa=\nabla \cdot \boldsymbol{n}=\nabla \cdot\left(\frac{\nabla \phi}{|\nabla \phi|}\right)=\frac{1}{|\nabla \phi|}\left[\nabla^{2} \phi-\frac{(\nabla \phi \cdot \nabla)|\nabla \phi|}{|\nabla \phi|}\right]
\end{equation}

Using the defined kernel, the terms in Eq. \ref{eq:curvature} can be expressed as:
\begin{equation} \label{eq:normal_terms}
|\nabla \phi|=-\frac{\partial \phi}{\partial n}=\frac{1-\phi^{2}}{\sqrt{2} W} \quad \text { and } \quad \frac{(\nabla \phi \cdot \nabla)|\nabla \phi|}{|\nabla \phi|}=\frac{\partial^{2} \phi}{\partial n^{2}}=-\frac{\phi\left(1-\phi^{2}\right)}{W^{2}}
\end{equation}

Substituting Eq. \ref{eq:normal_terms} into Eq. \ref{eq:curvature} yields to the following definition of interface curvature: 
\begin{equation}
\kappa=\frac{1}{|\nabla \phi|}\left[\nabla^{2} \phi+\frac{\phi\left(1-\phi^{2}\right)}{W^{2}}\right]
\end{equation}
which subsequently changes Eq. \ref{eq:advect_kappa} into:
\begin{equation} \label{eq:phase_field_semifnal}
\frac{\partial \phi}{\partial t}+a|\nabla \phi|+\boldsymbol{u}_{\mathrm{e}} \cdot \nabla \phi=b\left[\nabla^{2} \phi+\frac{\phi\left(1-\phi^{2}\right)}{W^{2}}\right]
\end{equation}

Eq. \ref{eq:phase_field_semifnal} is the the derived form of phase-field equation for tracking of an evolving interface, containing terms corresponding for normal interface motion, advection by an external field, and curvature-driven movement. The $|\nabla \phi|$ term in Eq. \ref{eq:phase_field_semifnal} can be replaced by its definition in Eq. \ref{eq:normal_terms} to form another version of the equation:

\begin{equation} \label{eq:phase_field_final}
\frac{\partial \phi}{\partial t}+a \frac{1-\phi^{2}}{\sqrt{2} W}+\boldsymbol{u}_{\mathrm{e}} \cdot \nabla \phi=b\left[\nabla^{2} \phi+\frac{\phi\left(1-\phi^{2}\right)}{W^{2}}\right]
\end{equation}
which is an easier version to be implemented using numerical methods. The unique term on the right-hand side of Eq. \ref{eq:phase_field_final} is a characteristic of the phase-field method.

\subsection{Level-set formulation}

From the mathematical perspective, the level set equation has a direct connection to the phase-field equation and can be derived by replacing the phase field variable by a sign distance function. To this end, Eq. \ref{eq:advect_kappa} can be rewritten to be a level set equation:
\begin{equation} \label{eq:ls_general}
\frac{\partial \psi}{\partial t}+a|\nabla \psi|+\boldsymbol{u}_{\mathrm{e}} \cdot \nabla \psi=b \kappa|\nabla \psi|,
\end{equation}
with $\psi$ being a sign distance function that descibes the distance of each point of the computational domain to the interface. This implies that the zero iso-contour of the level-set function defines the interface.

\section{Dimensionless forms for various cases}

\subsection{Stationary interface}

A stationary interface, where there is no interface motion ($a=0$ and $\boldsymbol{u}_{\mathrm{e}}=0$), is a primary problem to examine the formulation and select proper grid spacing parameters. For a 1-D case, Eq. \ref{eq:phase_field_final} can be simplified as:
\begin{equation} \label{eq:stationary_general}
\frac{\partial \phi}{\partial t}=b\left(\frac{\partial^{2} \phi}{\partial x^{2}}+\frac{\phi\left(1-\phi^{2}\right)}{w^{2}}\right)
\end{equation}

To scale this equation, the following dimensionless variables can be defined:
\begin{equation} \label{eq:dimenless_variables}
x^\prime = \frac{x}{x_c} \quad \text{and} \quad t^\prime = \frac{t}{t_c}
\end{equation}

So, Eq. \ref{eq:stationary_general} can be rewritten using these new variables:
\begin{equation} 
\frac{1}{t_{c}} \frac{\partial \phi}{\partial t^{\prime}}=\frac{b}{x_{c}^{2}} \frac{\partial^{2} \phi}{\partial x^{2}}+\frac{b}{w^{2}} f(\phi)
\end{equation}

Defining $x_c = w$ and $t_c=w^2/b$ leads to the following dimensionless form of Eq. \ref{eq:stationary_general}:

\begin{equation} \label{eq:stationary_dimenless}
\frac{\partial \phi}{\partial t^{\prime}}=\frac{\partial^{2} \phi}{\partial x^{\prime 2}}+\phi\left(1-\phi^{2}\right)
\end{equation}

Numerical results for Eq. \ref{eq:stationary_dimenless} is depicted in Fig.  \ref{fig:fig:pf_ls}, and the phase field profile is compared with the level set distance function profile. An appropriate value for grid spacing and the layer width should be selected such that $0.25 w < \Delta x^\prime < 0.5 w $. Additionally, the selected value of $w$ should satisfy $w < R/4.2$, in which $R$ is the local radius of curvature \cite{Sun2007}.

\begin{figure}
\medskip
\centering
\includegraphics[width=0.7\textwidth]{pf_ls.jpg}
\caption[Comparison of phase field variable and level set function]{Comparison of phase field variable and level set function on the interface of a stationary interface with step function like initial condition \cite{Sun2007}}
\label{fig:fig:pf_ls}
\end{figure}


\subsection{Evolution under constant normal speed}

For a problem in which the interface moves with constant velocity exclusively, Eq. \ref{eq:phase_field_final} can be simplified according to the condition of $a=\text{const.}$, $\boldsymbol{u}_{\mathrm{e}}=0$, and $b=0$ ($b$ exists in the equation as a numerical parameter for smoothing the interface and relaxation behavior of the phase-field profile):
\begin{equation}
\phi_{t}+a \frac{1-\phi^{2}}{\sqrt{2} w}=b \phi_{xx} + b\frac{f(\phi)}{w^{2}}
\end{equation}

Defining dimensionless variables according to Eq. \ref{eq:dimenless_variables} yields to:
\begin{equation}
\frac{1}{t_c}\phi_{t^{\prime}}+\frac{a}{w} \frac{1-\phi^{2}}{\sqrt{2}}=\frac{b}{w^2} \phi_{x^{\prime}x^{\prime}} + \frac{b}{w^{2}}f(\phi)
\end{equation}
which can be reordered to:
\begin{equation}
\phi_{t^{\prime}}+\frac{1-\phi^{2}}{\sqrt{2}}=b^{\prime}\left( \phi_{x^{\prime}x^{\prime}} + f(\phi)\right)
\end{equation}
with $b^{\prime}=b/aw$. For a stable numerical implementation, $\Delta t^{\prime}/\Delta x^{\prime} < 0.1$ and $b^{\prime} < 1.2$ should met roughly \cite{Sun2007}.


\subsection{Curvature-driven interface evolution}

A curvature-driven motion, which is desired for the current study, is straightforward to formulate using the phase-field method. In this specific case, $b$ is not a numerical parameter anymore. A dimensionless form of Eq. \ref{eq:phase_field_final} can be derived using a similar method for the stationary interface for a multidimensional case with $u_{\mathrm{n}}=-b\kappa$, $a=0$, and $\boldsymbol{u}_{\mathrm{e}}=0$:
\begin{equation} \label{eq:pf_curvature}
\frac{\partial \phi}{\partial t^{\prime}}=\nabla^{\prime 2} \phi+\phi\left(1-\phi^{2}\right)
\end{equation}
in which $t^{\prime}$ and $\nabla^{\prime}$ are defined similar to Eq. \ref{eq:dimenless_variables} as $t^{\prime}=t/(w^2/b)$ and $\nabla^{\prime}=\nabla/w$, respectively.

\section{Adapting the formulation for  curvature-driven tissue growth}

Neotissue growth in porous scaffolds has been shown to be depending on the local mean curvature of the interface between the scaffold and the neotissue \cite{Bidan2012, Bidan2012a, Rumpler2008}. Due to intrinsic support of interface curvature in phase-field and level-set methods, an \textit{in silico} model of curvature-based tissue growth can be implemented using these principles. The growth induced changes in the neotissue topology during the culture process can be seen as a moving interface between two different domains \cite{Rumpler2008}. In this study, one domain represents the neotissue volume  ($\Omega_{\text{nt}}$), and the other one is the void ($\Omega_{\text{v}}$), which are separated by an interface ($\Gamma$) as can be see in Fig. \ref{fig:growth_domain}.

\begin{figure}
\medskip
\centering
\includegraphics[width=0.5\textwidth]{growth_domain.jpg}
\caption[Schematic representation of the phase-field and level-set models for tissue growth]{Schematic representation of the phase-field and level-set models for tissue growth}
\label{fig:growth_domain}
\end{figure}

The interface $\omega$ evolves over time and fill the void space, which has a faster growth in regions with higher curvature. Based on this definition, the phase-field function can be defined as follows to separate these domains:
\begin{equation} \label{eq:pf_domains}
\left\{\begin{array}{ll}
\phi = 1 & \text{ in } \Omega_{\text{nt}} \\
\phi = -1 & \text{ in } \Omega_{\text{v}} \\
-1 < \phi < 1 & \text{ in } \Gamma
\end{array}\right.
\end{equation}

Similarly, a level-set function ca be defined such that it separates the neotissue and void domains:
\begin{equation} \label{eq:ls_domains}
\left\{\begin{array}{ll}
\psi > 0 & \text{ in } \Omega_{\text{nt}} \\
\psi < 0 & \text{ in } \Omega_{\text{v}} \\
\psi = 0 & \text{ in } \Gamma
\end{array}\right.
\end{equation}

In order to adapt Eq. \ref{eq:pf_curvature} for the curvature-driven process of neo-tissue growth, one can consider the following equation:
\begin{equation} \label{eq:pf_tissue}
\frac{\partial \phi}{\partial t^{\prime}}=\left(\nabla^{\prime 2} \phi+\phi\left(1-\phi^{2}\right)\right).H\left(\nabla^{\prime 2} \phi+\phi\left(1-\phi^{2}\right)>0\right)
\end{equation}
in which $H$ denotes a heaviside step function. Eq. \ref{eq:pf_tissue} implies that the growth is only allowed for regions with a positive curvature (right hand side of Eqs. \ref{eq:phase_field_final} and \ref{eq:pf_curvature}).

Using the same approach, a similar level set formulation can be obtained based on Eq. \ref{eq:ls_general} by omitting the normal velocity and curvature terms and embedding the effect of the curvature in the velocity field. Doing this yields to a convection equation for the distance function:
\begin{equation} \label{eq:ls_advect}
\frac{\partial \psi}{\partial t}+\boldsymbol{u} \cdot \nabla \psi=0,
\end{equation}
in which the convection velocity field can be defined as:
\begin{equation} \label{eq:ls_veloc}
\boldsymbol{u}=\left\{\begin{array}{ll}
-\kappa \boldsymbol{n} & \text { if } \kappa>0 \\
0 & \text { if } \kappa \leq 0
\end{array}\right.
\end{equation}
with $\kappa$ being calculated similar to Eq. \ref{eq:curvature} for a distance function $\psi$, implying that neotissue grows faster where the curvature is higher and does not grow if the curvature is negative or equal to zero \cite{Bidan2012}. The negative sign in Eq. \ref{eq:ls_veloc} is due to to our definition of $\psi$, where the normal $n_\Gamma$  points toward neotissue, so growth has to be towards the opposite of the gradient of the level-set function ($ \nabla \psi$).

\section{Numerical implementation}

\subsection{Phase-field model}

Numerical solution of the phase-field equation involves dealing with the non-linearity of the equation. Additionally, in the case of dimensional form (Eq. \ref{eq:phase_field_final}), small coefficients of the state variable in the PDE leads to numerical difficulties. As a result, numerical implementation of the phase-field equation, especially for the spectral methods such as the finite element method, is still tricky and an active field of research \cite{Shen2010,Abboud2019}. 

In the finite element method, the solution of a PDE is calculated based on a sum of a set of certain basis functions, which are commonly piecewise polynomial functions that are non-zero only on a small element. For doing this, the PDE is first written in a weak formulation, and then the weak form is projected on a discretized space (a set of elements) to be written as the summation of the basis functions. 

In this section, the numerical solution of the stationary form (Eq. \ref{eq:stationary_general}) and its corresponding considerations are elaborated as an example of employing the finite element formulation for simulating the phase-field equation. So, by assuming $b=1$ and $f(\phi)=\phi\left(1-\phi^{2}\right)$, the problem can be summarized as:
\begin{equation} \label{eq:fe_problem}
\left\{\begin{array}{ll}
\frac{\partial \phi}{\partial t}-\Delta \phi+\frac{1}{w^{2}} f(\phi)=0, & (x, t) \in \Omega \times(0, T] \\
\left.\frac{\partial \phi}{\partial n} \right|_{\partial \Omega}=0 \\
\left.u\right|_{t=0}=u_{0}
\end{array}\right.
\end{equation}
which demonstrates the PDE, the boundary condition, and the initial condition of the phase field variable where $\Omega$ is the domain of interest, $\partial \Omega$ is its boundary, and $T$ is the final time. Deriving the weak formulation of Eq. \ref{eq:fe_problem} is relatively straightforward as it can be seen as a time-dependent diffusion-reaction PDE, but the difficulty arises for choosing the numerical stability scheme for discretizing the temporal derivative and dealing with the non-linearity of $f(\phi)$ when normally the $\frac{1}{w^{2}}$ coefficient is a small number.

Incorporating a first-order semi-explicit scheme for Eq. \ref{eq:fe_problem} yields to \cite{Abboud2019}:
\begin{equation} \label{eq:semi-implicit}
\frac{1}{\Delta t}\left(\phi^{n+1}-\phi^{n}, v\right)+\left(\nabla \phi^{n+1}, \nabla v\right)+\frac{1}{w^{2}}\left(f\left(\phi^{n}\right), v\right)=0, \quad \forall v \in H^{1}(\Omega)
\end{equation}
where $\Delta t$ is the time step, $(\cdot,\cdot)$ denotes the inner product, and $H^{1}(\Omega)$ is the Sobolev space of the domain $\Omega$, which is a space of functions whose derivatives are square-integrable functions in $\Omega$. The main issue with this discretization scheme is its restrictive time step condition which should satisfy \cite{Shen2010}:
\begin{equation}
\Delta t < \frac{2w^2}{L}
\end{equation}
where $L$ is a limit related to the non-linear part:
\begin{equation}
\max \left|f^{\prime}(\phi)\right| \leq L
\end{equation}
Obviously, since $\Delta t \sim w^2$, a very small time step is required to achieve stability in this scheme. 

Taking advantage of a fully implicit scheme improves the stability because it will be unconditionally stable, but it results in an equation that is difficult to implement as it needs solving a fixed point problem at each step. For example, a modified second-order implicit Crank-Nicolson scheme for Eq. \ref{eq:fe_problem} can be written as \cite{Abboud2019,Elliott1989}:
\begin{equation}
\left(\frac{\phi^{n+1}-\phi^{n}}{\Delta t}, v\right)+\left(\nabla \frac{\phi^{n+1}+\phi^{n}}{2}, \nabla v\right)+\frac{1}{w^{2}}\left(\tilde{f}\left(\phi^{n+1}, \phi^{n}\right), v\right)=0, \quad \forall v \in H^{1}
\end{equation}
where:
\begin{equation}
\tilde{f}(u, v)=\left\{\begin{array}{ll}
\frac{F(u)-F(v)}{u-v} & \text { if } u \neq v \\
f(u) & \text { if } u=v
\end{array}\right.
\end{equation}
in which $F$ is the potential term ($f(\phi)=F^{\prime}(\phi)$).

An alternative can be deriving a stabilized semi-implicit scheme by adding a stabilization term to Eq. \ref{eq:semi-implicit}. A first-order version of such a scheme can be written as:
\begin{equation}
\left(\frac{1}{\Delta t}+\frac{S}{w^{2}}\right)\left(\phi^{n+1}-\phi^{n}, v\right)+\left(\nabla \phi^{n+1}, \nabla v\right)+\frac{1}{w^{2}}\left(f\left(\phi^{n}\right), v\right)=0, \quad \forall v \in H^{1}(\Omega)
\end{equation}
which is unconditionally stable for any $S \geq \frac{L}{2}$ \cite{Shen2010}.


\subsection{Level-set model}

The derived level-set PDE (Eq. \ref{eq:ls_advect}) is an advection equation, which can be implemented numerically using the finite element method, in which the temporal term is discretized by the backward Euler method, and the advection term can be treated with the method of characteristics. 

A key parameter of the developed model is the local growth velocity of the neotissue. In the current implementation, this space dependent velocity was only depending on the local mean curvature of the interface as has been shown in \cite{Bidan2012a,Guyot2014}. In order to match the growth velocity to experimental data, a coefficient can be added to the derived interface convection velocity (Eq. \ref{eq:ls_veloc}):
\begin{equation} \label{eq:ls_veloc2}
\boldsymbol{u}=\left\{\begin{array}{ll}
-\kappa A \boldsymbol{n} & \text { if } \kappa>0 \\
0 & \text { if } \kappa \leq 0
\end{array}\right.
\end{equation}

The model calibration performed in Guyot et al. \cite{Guyot2014} was for titanium scaffolds culture in a bioreactor setting, estimating parameter $A$ to be $4\times10^{-14} \text{m}^2/\text{s}$, obtained using trial and error from the experimental data on low flow rate tests \cite{Papantoniou2014}.
More dedicated calibration experiments are being performed on prismatic structures, demonstrating a considerably slower growth on the CaP scaffolds, nevertheless confirming the curvature-based nature of tissue growth.

In practical implementations, the distance function is not differentiable at every location of the domain due to discontinuities in the gradients, so one can consider taking advantage of artificial diffusion terms to overcome this issue, leading to the following equations for the normal vector and curvature calculation:
\begin{equation}
\boldsymbol{n}=\frac{\nabla \varphi}{|\nabla \varphi|}+\varepsilon \Delta \boldsymbol{n}
\end{equation}
\begin{equation}
\kappa=\nabla \cdot \boldsymbol{n}+\varepsilon \Delta \kappa
\end{equation}
in which $\varepsilon$ denotes the numerical diffusion coefficient.


\subsection{Parallel computing}


\section{Simultaion setup}

%The initial configuration of the distance function φ corresponds to a homogenous single cell layer over the scaffold struts with a thickness equal to 20 µm (Darling and Guilak 2008). 

%Neotissue growth was then simulated for a variety of lattice-based structures as well as structures from the triply periodic minimal surface (TPMS) family and compared in a qualitative way. A full quantitative prediction is not possible due to the absence of relevant validation experiments, which explains why comparisons between geometries are made over non-dimensional time.


\section{Results and Discussion}


\section{Challenges in coupling tissue growth and biodegradation models}

\begin{subappendices}

\section{Implementing the phase-field model using physics-informed neural networks}



\end{subappendices}

%%%%%%%%%%%%%%%%%%%%%%%%%%%%%%%%%%%%%%%%%%%%%%%%%%
% Keep the following \cleardoublepage at the end of this file, 
% otherwise \includeonly includes empty pages.
\cleardoublepage

