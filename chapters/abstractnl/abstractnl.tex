% !TeX root = ../../thesis.tex
\chapter{Beknopte samenvatting}

Degradeerbare metalen worden bij een brede set toepassingen steeds populairder. Voor toepassingen in het biomedisch velt, zijn biocompatibiliteit, biologische afbreekbaarheid en het bevorderden van biologische processen kritische eigenschappen die bepalen of een materiaal toepasbaar kan zijn. Enkele metalen die aan deze criteria aan voldoen, en hierdoor tot de bioafbreekbare metallische biomaterialen behoren, zijn magnesium (Mg), ijzer (Fe) en zink (Zn). Dankzij hun mechanische eigenschappen, zijn deze metallische biomaterialen geschikt voor toepassingen waarin hogere belastingen kunnen worden verwacht. Dit is onder anderen het geval in botgenezing en cardiovasculaire toepassingen. Bij zulke toepassingen, kunnen biologisch afbreekbare metalen een tijdelijke behoefte aan mechanische ondersteuning bieden terwijl spanningsafscherming (bij orthopedische toepassingen) op de lange termijn wordt vermeden en daarmee ook de noodzaak voor een revisiechirurgie, zoals vaak nodig bij gebruik van niet degradeerbare materialen. Ondanks de voordelen van biologische afbreekbare metalen, zijn er ook enkele nadelen zoals hun snelle degradatie en ongecontroleerde ionenafgifte welke tot uitdagingen kunnen leiden bij praktische toepassingen. Naast de mogelijkheid om experimenteel de eigenschappen van biologisch afbreekbare (poreuze) metalen te onderzoeken, kan het degradatie proces ook met computer modellen worden gesimuleerd wat een efficiënte en kost effectieve manier bied om de volgende generatie medische apparaten en implantaten te ontwerpen. Zo een gevalideerd computermodel van het degradatie proces kan, onder anderen, het afstellen van de degradatie parameters en het optimaliseren van het ontwerp voor specifieke toepassingen vergemakkelijken.

In deze studie, is zowel een wiskundig als computationeel model ontwikkeld dat in staat is het biodegradatiegedrag van biologisch afbreekbare metallische biomaterialen te voorspelen. Dit model focust zich op het gedrag van Mg en houd rekening met het vrijkomen van metaalionen, de veranderingen in pH, de vorming van een beschermende film, het effect van verschillende ionen in de omgeving en het effect van de perfusie van het omringende vloeistof, indien van toepassing. Dit is bereikt door een stelsel van tijdsafhankelijke reactie-diffusie-convectie partiële differentiaalvergelijkingen af te leiden uit de onderliggende oxidatie-reductiereacties en deze op te lossen met behulp van de eindige-elementenmethode. Het level-set formalisme is gebruikt om de interface tussen het degraderende biomateriaal en zijn omgeving te volgen tijdens het degradatie proces. Het model is gevalideerd door de in silico voorspelde globale en lokale pH-veranderingen te vergelijken met experimenteel verkregen resultaten van corrosietesten.

Het volgen van het bewegend front op de diffusie interface vereist dat de diffusie toestandsvariabelen met een hoge numerieke nauwkeurigheid worden bepaalt. Voor het bepalen van deze variabelen met de nodige nauwkeurigheid is een fijne mesh nodig, wat leidt tot rekenintensieve simulaties. Om desondanks de rekenintensiteit in staat te zijn interactieve simulaties te leveren die in een redelijke tijd opgelost kunnen worden, zijn schaalbare parallellisatie technieken geïmplementeerd dat het mogelijk maken om het model op massaal parallelle systemen op te lossen, wat de simulatietijd aanzienlijk verkort. Vervolgens, is het schaalgedrag van de modellen geëvalueerd met honderden tot duizenden CPU-cores in een high-performance computeromgevingen. Ook is het degradatie model gekoppeld aan vloeistofstroommodellen zodat het effect van de hydrodynamica en perfusiecondities kon worden vast gelegd. Tot slot, is het model gebruikt in verschillende multi-fysische scenario's om de mogelijkheden omtrent het integreren van het model in andere workflows binnen de biomedische ingenieurswetenschappen aan te tonen.

%%%%%%%%%%%%%%%%%%%%%%%%%%%%%%%%%%%%%%%%%%%%%%%%%%
% Keep the following \cleardoublepage at the end of this file, 
% otherwise \includeonly includes empty pages.
\cleardoublepage

% vim: tw=70 nocindent expandtab foldmethod=marker foldmarker={{{}{,}{}}}
